%______________________________________________________________________________________________________________________
% @brief    LaTeX2e Resume for Kathryn D. Huff

\documentclass[margin,line]{resume}
\usepackage{bibentry}
\usepackage{url}
\usepackage{xspace}
\usepackage{graphicx}
\newcommand{\Cyclus}{\textsc{Cyclus}\xspace}%


%______________________________________________________________________________________________________________________
\begin{document}
\name{\Large Kathryn D. Huff}
\begin{resume}

    % Contact Information
    \section{\mysidestyle Contact\\Information}
    Blue Waters Assistant Professor \hfill mobile: (281) 734-1342 \vspace{0mm}\\\vspace{0mm}%
        \textsl{University of Illinois, Urbana-Champaign}
        \hfill e-mail: kdhuff@illinois.edu            \vspace{0mm}\\\vspace{0mm}%
    \textsl{Nuclear, Plasma, and Radiological Engineering}
        \hfill website: arfc.npre.illinois.edu     \vspace{0mm}\\\vspace{-4.5mm}%
    \textsl{Affiliate Faculty, National Center for Supercomputing Applications}           \vspace{0mm}\\\vspace{0mm}%

    %__________________________________________________________________________________________________________________
    % Resume Objective
    %\section{\mysidestyle Objective}
		%Seeking research and teaching opportunities in nuclear engineering and scientific computation.%
    %__________________________________________________________________________________________________________________
    % Research Interests
    %\section{\mysidestyle Research\\Interests}
    %Advanced nuclear reactors and fuel cycles, multi-physics
    %            simulation, nuclear fuel cycle analysis,
    %            scientific computation.
                %
    %__________________________________________________________________________________________________________________
    % Academic Appointments
    %	 	\vspace{-2mm}
    %\section{\mysidestyle Academic\\Appointments}
    %           \vspace{-2mm}\\\vspace{-3mm}%
    %__________________________________________________________________________________________________________________
    % Advanced Reactors
    \section{\mysidestyle Fuel\\Cycle\\Analysis\\ \vspace{12mm}
        \includegraphics[width=0.8\textwidth]{fc}}
Our nuclear energy future may involve a mixture of technologies, reprocessing 
        schemes, and waste management strategies. Deployments may be driven by 
        politics or demand, the options may be constrained by complex 
        logistics, and the assessment of impacts requires analysis at scale. 
        Our work focuses on modeling, simulation, and analysis of the global 
        nuclear fuel cycle, with an emphasis on sustainability.
        \vspace{2mm}

      \begin{bibenum}
       \item \textbf{Huff,~K.}, Gidden,~M., Carlsen,~R., Flanagan,~R.,
       McGarry,~M., Opotowsky,~A., Schneider,~E.,
       Scopatz,~A., Wilson,~P.  ``Fundamental Concepts in the \Cyclus Nuclear Fuel Cycle
       Simulation Framework.'' \textbf{Advances in Engineering Software}, vol. 94, pp. 46–59, Apr. 2016.
      \item \textbf{Huff, K.}, Bae, J., Mummah, K., Flanagan, R., Scopatz, A.
            ``Current Status of Predictive Transition Capability in Fuel Cycle 
            Simulation'' \textbf{GLOBAL 2017 International Nuclear Fuel Cycle 
            Conference}, Seoul, South Korea. September 2017. \emph{(submitted)}
      \item Bae, J., Roy, W., \textbf{Huff, K.}.
            ``Benefits of Siting a Borehole Repository on Non-Operating Nuclear 
            Facility'' Paper 19727.  \textbf{International High-Level Radioactive 
            Waste Management Converence (IHLRWM 2017)},
            Charlotte, NC. April 2017. 
      \item Greenberg, H., Fratoni, M., Djokic, D., \textbf{Huff, K.},
         Nibbelink, R., Scopatz, A. ``The Application of CYCLUS to Fuel Cycle
         Transition Modeling'' Paper 5061.
         \textbf{Proceedings of Global}, Paris, France. September 2015.
      \end{bibenum}


      \vspace{2mm}
    %__________________________________________________________________________________________________________________
    % Computing
        \section{\mysidestyle Coupled\\Physics of\\Advanced\\Reactors\\\vspace{4mm}
        \includegraphics[width=0.8\textwidth]{pbfhr}}
Advanced nuclear reactors often involve complex geometries, unique materials,
and new physical regimes. To simulate the coupled physics in such designs,
current tools must be extended to capture those geometries, materials, and
regimes. Our work focuses on extending current tools with features essential to
simulating advanced reactor multiphysics.
        \vspace{2mm}

      \begin{bibenum}
      \item Andreades,~C., Cisneros,~A.T., Choi,~J.K., Chong,~A.Y., 
              Fratoni,~M., Hong,~S., Huddar,~L.R., \textbf{Huff,~K.}, Kendrick,~J., 
              Krumwiede,~D.L., Laufer,~M., Munk,~M., Scarlat,~R.O., Wang,~X., 
              Zwiebaum,~N., Greenspan,~E. and P. Peterson.  ``Design Summary of 
              the Mark-I Pebble-Bed, Fluoride Salt–Cooled, High-Temperature 
              Reactor Commercial Power Plant,'' \textbf{Nuclear Technology}, vol. 195, 
              no. 3, pp. 222-238, Sep. 2016.
      \item Wang, X., \textbf{Huff, K.}, Aufiero, M., Peterson, P., Fratoni, M.
            ``Coupled reactor kinetics and heat transfer model for nuclear 
            reactor transient analysis.'' Paper 60728.  \textbf{24th 
            International Conference on Nuclear Engineering (ICONE24)}, 
            Charlotte, NC.  June 2016. 
      \item Wang, X., \textbf{Huff, K.}, Aufiero, M., Peterson, P., Fratoni, M.
            ``A sensitivity study of a coupled kinetics and thermal-hydraulics 
            model for Fluoride-salt-cooled, High-temperature Reactor (FHR) 
            transient analysis.'' \textbf{The International 
            Congress on Advances in Nuclear Power Plants (ICAPP)}, San 
            Francisco, CA. April 2016.
      %\item \textbf{Huff, K.}, ``PyRK: Python for Reactor Kinetics.''
      %   \textbf{Proceedings of the 14th Python in Science Conference}, Austin,
      %   TX. July 2015.
      \end{bibenum}


      \vspace{2mm}
    %__________________________________________________________________________________________________________________
    % Fuel Cycles
    \section{\mysidestyle Advanced\\ Computation\\ \vspace{16mm}
        \includegraphics[width=0.8\textwidth]{bw}}
 A crosscutting theme of our research is an emphasis on advancing methods and 
        software for computational nuclear engineering. Simulations of reactors 
        and fuel cycle systems are sufficiently complex that sophisticated 
        scientific software and high-performance computing resources are 
        essential to understanding and improving them. Accordingly, the 
        Advanced Reactors and Fuel Cycles group is proud to be affiliated with 
        the University of Illinois National Center for Supercomputing 
        Applications and its Blue Waters computing facility.

        \vspace{2mm}

      \begin{bibenum}
      \item Scopatz, A., \textbf{Huff, K.}. ``Effective Computation in
      Physics: Field Guide to Research in Python'' O'Reilly Media. 2015.
      \url{shop.oreilly.com/product/0636920033424.do}.
       \item Aruliah,~D.A., Brown,~C.T., Chue Hong,~N.P., Davis,~M., Guy,~R.T.,
          Haddock,~S.H.D., \textbf{Huff,~K.}, Mitchell,~I., Plumbley,~M., Waugh,~B.,
          White,~E.P., Wilson,~G.V., and Wilson,~P.P.H.  ``Best Practices For
          Scientific Computing.'' \textbf{PLOS Biology}, Vol 1, Issue 12,
          2014. \url{dx.doi.org/10.1371/journal.pbio.1001745}
      %\item \textbf{Huff, K.}. ``Case Study: Cyclus Project,'' in The Practice 
      %        of Reproducible Research, 1st ed., Justin Kitzes, Fatma Imamoglu, 
      %        and Daniel Turek, Eds. University of California, Berkeley: 
      %        University of California Press. 2017.
      %\item \textbf{Huff, K.}. ``Lessons Learned,'' in The Practice of Reproducible 
      %        Research, 1st ed., Justin Kitzes, Fatma Imamoglu, and Daniel 
      %        Turek, Eds. University of California, Berkeley: University of 
      %        California Press. 2017.
      \end{bibenum}


    %__________________________________________________________________________________________________________________
\end{resume}

\end{document}


%______________________________________________________________________________________________________________________
% EOF

